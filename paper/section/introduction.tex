\section{Introduction and Motivation}

Cartograms are representations of geographical space generated by a value-by-area mapping technique that combines statistical and geographical information \cite{dent2009Cartography}. \new{Various styles of cartograms have been proposed and implemented, covering topics on urban planning \cite{harris2018Mapping, arranz-lopez2021Enduser}, natural hazard forecasting \cite{pappenberger2019Cartograms, park2020Flood}, conservation and environmental planning \cite{galluzzi2018Mapping, rocchini2019Cartogramming}, political and social demographics \cite{breitzman2018Using, alieva2021How}, and public health decision making \cite{gao2020Visualising, sack2021Visualizing}.}

In this paper, we follow the categorization of cartograms by \citea{nusrat2016State}: contiguous, non-contiguous, rectangular, and Dorling. Among these four types, trade-offs between accuracies are made (See \tableref{table:accuracy}). Building on Demers cartograms \cite{ian2002Cartogram}, we develop dynamic topological features, such as rivers, aiming to improve the readability and geographical accuracy without sacrificing statistical accuracy. We incorporate a cartographic layout algorithm that includes these topological features to update the layout of the nodes representing geographical regions. To minimize geographical errors, the algorithm also updates the position of rivers to accommodate the node layout. We then apply the algorithm a real world Electronic Health Records (EHR) dataset to assess the accuracy of our algorithm.

\new{Our contributions include:}

\begin{itemize}
    \item A new variant of Demers cartograms that incorporates dynamic topological features to improve readability and recognizability.
    \item A novel layout algorithm that preserves node positions relative to dynamic topological features such as rivers.
\end{itemize}