\section{Introduction and Motivation}

Cartograms are representations of hybrid geographical and data space based on a value-by-area mapping technique that combines statistical and geographical information \cite{dent2009Cartography}. Various styles of cartograms have been proposed and implemented, covering applications such as urban planning \cite{harris2018Mapping, arranz-lopez2021Enduser}, natural hazard forecasting \cite{pappenberger2019Cartograms, park2020Flood}, conservation and environmental planning \cite{galluzzi2018Mapping, rocchini2019Cartogramming}, political and social demographics \cite{breitzman2018Using, alieva2021How}, and public health decision making \cite{gao2020Visualising, sack2021Visualizing}.

In this paper, we follow the categorization of cartograms by \citea{nusrat2016State}: contiguous, non-contiguous, rectangular, and Dorling. Among these four types, a trade-off between accuracy is made (See \tableref{table:accuracy}). For this project we focus on non-contiguous cartograms like Demers cartograms, because they facilitate statistical comparison between regions and they can make good use of screen space. Building on Demers cartograms \cite{ian2002Cartogram}, we introduce and develop novel dynamic topological features, such as rivers, aiming to improve the readability and geographical accuracy without sacrificing statistical accuracy. We implement a cartographic layout algorithm that includes these topological features into the layout of the nodes representing geographical regions. To minimize geographical errors and make efficient use of screen space, the algorithm also updates the position of rivers to accommodate the node layout. We then apply the algorithm a real world case-study using an Electronic Health Records (EHR) dataset to assess the accuracy of our algorithm.

\new{Our contributions include:}

\begin{itemize}
    \item A new variant of Demers cartograms that incorporates dynamic topological features to improve readability and recognizability.
    \item A novel layout algorithm that preserves node positions relative to dynamic topological features such as rivers.
    \item A user study evaluation of the technique with an application to EHRs
\end{itemize}