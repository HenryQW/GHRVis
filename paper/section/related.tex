\section{Related Work}

This section introduces the definition of cartogram types, describes relevant applications of cartograms, and provides a brief overview of some real-world applications of cartogram-based visualizations.

\subsection{Definitions}

Cartograms combine statistical and geographical information in thematic maps. We follow the categorization of cartograms by \citea{vankreveld2004Rectangular}: contiguous, non-contiguous, rectangular, and Dorling. \citea{nusrat2016State} summarize three major accuracy dimensions for cartograms: statistical, geographical, and topological. Each cartogram design may make accuracy trade-offs between dimensions, we provide an overview in Table \ref{table:accuracy}.

{
\renewcommand{\arraystretch}{1.5}
\begin{table}[!b]
	\centering
	\resizebox{\columnwidth}{!}{
		\begin{tabulary}{\columnwidth}{|*{5}{l|}}
			\hhline{~|*{3}{-}|~}
			\multicolumn{1}{c|}{} & \multicolumn{3}{c|}{\cellcolor{Mycolor2} \textbf{Accuracy}} \\
			\hhline{~|*{4}{-}}
			\multicolumn{1}{c|}{\textbf{Catogram Type}} &
			\textbf{Statistical} &
			\textbf{Geographical} &
			\textbf{Topological} &
			\textbf{Contiguity} \\
			\hline
			Contiguous & \cellcolor{Mycolor3}Variable & \cellcolor{Mycolor3}Variable & Accurate & Yes \\
			\hline
			Non-contiguous & \cellcolor{Mycolor5}\textcolor{white}{Accurate} & \cellcolor{Mycolor5}\textcolor{white}{Shape is accurate} &\cellcolor{Mycolor6}Inaccurate & No   \\
			\hline
			Rectangular & \cellcolor{Mycolor3}Variable & Shape is inaccurate & \cellcolor{Mycolor3}Variable & Yes   \\
			\hline
			Dorling & Accurate & Inaccurate & Inaccurate & No  \\
			\hline
		\end{tabulary}
	}
	\caption{\textcolor{Mycolor3}{\textbf{Trade-off between dimensions}}. \textcolor{Mycolor6}{\textbf{Dimension sacrificed}} in order to improve \textcolor{Mycolor5}{\textbf{target dimension}}'s accuracy.}
	\label{table:accuracy}
\end{table}
}


Dorling cartograms are non-contiguous and do not preserve geography and topology. A Dorling cartogram is statistically accurate, regions are represented by circles and the statistic of interest is represented by the circle size \cite{dorling2011Area}. 
In Demers Cartograms, a variant of Dorling, squares are used instead to capture a certain level of topology \cite{cano2015Mosaic}. Dorling cartograms are unable to maintain topological accuracy as circles are often repositioned to remove overlaps. 

Rectangular cartograms are contiguous and do not preserve geographical accuracy \cite{raisz1934Rectangular}. Depending on the variant, a rectangular cartogram may trade between statistical and topological accuracy.

Mosaic cartograms are contiguous and sacrifice statistical accuracy to preserve some level of geographical accuracy \cite{cano2015Mosaic}. Some variants are able to preserve topological accuracy as well.

\subsection{Peer-reviewed Applications}

\citea{warf2008Geography} use a Dorling cartogram to represent the religious diversity in the United States. \citea{sun2010Effectiveness} visualize 1996 US election data and 2005 China population date using Dorling, Mosaic, and contiguous cartograms. \citea{cruz2017Adapted} adapts a Dorling cartogram with both contiguous and non-contiguous cartograms to represent the gender pay gap in Portugal. \citea{gao2020Visualising} present a Dorling cartogram to illustrate COVID-19 infections in China. \citea{tong2018cartograms} use a Demers cartogram to visualize health-related date by regions in England, the work introduces a novel technique to remove overlaps of squares based on topological features, aiming to improve both geographical and topological accuracy. \citea{nusrat2020Recognition} investigate the memorability of contiguous and Dorling cartograms by using multiple datasets including demographics, agriculture, and retail data in the US. See Table \ref{table:region vs node} for a list of literature that adopts cartograms for visualization with corresponding geographical regions and node counts.


{
\renewcommand{\arraystretch}{1.5}
\begin{table*}[!tb]
	\centering
	\resizebox{\textwidth}{!}{
		\begin{tabulary}{\textwidth}{|*{4}{l|}r|c|}
			\hhline{~|*{4}{-}|~}
			\multicolumn{1}{c|}{\textbf{Literature}} &
			\textbf{Title} &
			\textbf{Cartogram Type(s)} &
			\textbf{Geographic Region(s)} &
			\textbf{Number of Nodes} &
			\multicolumn{1}{c}{\textbf{Year}} \\
			\hline
			
			% \citea{auber2007Geographical} & \citetitlea{auber2007Geographical} & US & 19 & 2007 \\
			% \hline
			\citea{warf2008Geography} & \citetitlea{warf2008Geography} & Dorling & US & 3,142 & \citeyear{warf2008Geography} \\
			\hline
			\citea{sun2010Effectiveness} & \citetitlea{sun2010Effectiveness} & Dorling, Mosaic, Contiguous & US, China & 34 - 49 & \citeyear{sun2010Effectiveness} \\
			\hline
			\citea{cruz2017Adapted} & \citetitlea{cruz2017Adapted} & Dorling, Non-contiguous, Contiguous & Portugal & 2,882 & \citeyear{cruz2017Adapted} \\
			\hline
			\citea{tong2018cartograms} & \citetitlea{tong2018cartograms} & Demers & England & 209 & \citeyear{tong2018cartograms} \\
			\hline
			\citea{gao2020Visualising} & \citetitlea{gao2020Visualising} & Dorling & China & 34 & \citeyear{gao2020Visualising} \\
			\hline
			\citea{nusrat2020Recognition} & \citetitlea{nusrat2020Recognition} & Contiguous, Dorling & Portugal & 49 & \citeyear{nusrat2020Recognition} \\
			\hline
			
			% \multicolumn{1}{c|}{\textbf{Total unique papers: 51}} & 24&15&12& \multicolumn{1}{c}{}  \\
			% \hhline{~|*{3}{-}|~}

		\end{tabulary}
	}
	\caption{}
	\label{table:region vs node}
\end{table*}
}

\subsection{Other Applications}

Cartogram has become a popular choice of visualization in covering various topics by the media, in order to present an unbiased and complete picture of the news. \citea{newman2016Election} generates contiguous cartograms to present US Election results in 2012 and 2016. The Washington Post adopts cartograms to visualize the US overseas economic assistance, arm sales (Mosaic) \cite{bearak2016Everything}, the 2016 US Election (contiguous) \cite{gamio2016Election}, and the Brexit Referendum (Mosaic) \cite{taylor2016What}. The National Geographic uses contiguous and Mosaic cartograms to analyze the 2016 US Election results \cite{miller2016Election}, the same topic is also covered by the Financial Times with a Dorling cartogram \cite{stabe2016Search}. \citea{sandberg2018Cartogram} reports the 2018 US midterm Election with a Mosaic cartogram, the same approach is used to cover the 2020 US Election by the New York Times \cite{thelearningnetwork2020What} and Bloomberg \cite{mccartney20202020}.